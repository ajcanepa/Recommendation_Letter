%%%%%%%%%%%%%%%%%%%%%%%%%%%%%%%%%%%%%%%%%
% Professional Formal Letter
% LaTeX Template
% Version 1.0 (28/12/13)
%
% This template has been downloaded from:
% http://www.LaTeXTemplates.com
%
% Original author:
% Brian Moses (http://www.ms.uky.edu/~math/Resources/Templates/LaTeX/)
% with extensive modifications by Vel (vel@latextemplates.com)
%
% License:
% CC BY-NC-SA 3.0 (http://creativecommons.org/licenses/by-nc-sa/3.0/)
%
%%%%%%%%%%%%%%%%%%%%%%%%%%%%%%%%%%%%%%%%%

%----------------------------------------------------------------------------------------
%	PACKAGES AND OTHER DOCUMENT CONFIGURATIONS
%----------------------------------------------------------------------------------------

\documentclass[11pt,a4paper]{letter} % Specify the font size (10pt, 11pt and 12pt) and paper size (letterpaper, a4paper, etc)

\usepackage{graphicx} % Required for including pictures
\usepackage{microtype} % Improves typography
%\usepackage{gfsdidot} % Use the GFS Didot font: http://www.tug.dk/FontCatalogue/gfsdidot/
%\usepackage[T1]{fontenc} % Required for accented characters
\usepackage[utf8]{inputenc}
\usepackage{hyperref}
\usepackage[spanish]{babel}

% Create a new command for the horizontal rule in the document which allows thickness specification
\makeatletter
\def\vhrulefill#1{\leavevmode\leaders\hrule\@height#1\hfill \kern\z@}
\makeatother

%----------------------------------------------------------------------------------------
%	DOCUMENT MARGINS
%----------------------------------------------------------------------------------------

\textwidth 6.75in
\textheight 9.25in
\oddsidemargin -.25in
\evensidemargin -.25in
\topmargin -1in
\longindentation 0.50\textwidth
\parindent 0.4in

%----------------------------------------------------------------------------------------
%	SENDER INFORMATION
%----------------------------------------------------------------------------------------

\def\Who{Álvar Arnaiz González} % Your name
\def\What{, Coordinador del GII} % Your title
\def\Where{Área de Lenguajes y \\Sistemas Informáticos} % Your department/institution
\def\Address{Escuela Politécnica Superior\\Avda. Cantabria s/n} % Your address
\def\CityZip{Burgos 09006} % Your city, zip code, country, etc
\def\Email{E-mail: alvarag@ubu.es} % Your email address
\def\TEL{Phone: 617 652 369} % Your phone number
\def\URL{URL: http://admirable-ubu.es/} % Your URL

%----------------------------------------------------------------------------------------
%	HEADER AND FROM ADDRESS STRUCTURE
%----------------------------------------------------------------------------------------

\address{
\includegraphics[width=1in]{logo} % Include the logo of your institution
\hspace{5.1in} % Position of the institution logo, increase to move left, decrease to move right
\vskip -1.07in~\\ % Position of the text in relation to the institution logo, increase to move down, decrease to move up
\Large\hspace{1.5in}UNIVERSIDAD \hfill ~\\[0.05in] % First line of institution name, adjust hspace if your logo is wide
\hspace{1.5in}DE BURGOS \hfill \normalsize % Second line of institution name, adjust hspace if your logo is wide
\makebox[0ex][r]{\bf \Who \What }\hspace{0.08in} % Print your name and title with a little whitespace to the right
~\\[-0.11in] % Reduce the whitespace above the horizontal rule
\hspace{1.5in}\vhrulefill{1pt} \\ % Horizontal rule, adjust hspace if your logo is wide and \vhrulefill for the thickness of the rule
\hspace{\fill}\parbox[t]{2.85in}{ % Create a box for your details underneath the horizontal rule on the right
\footnotesize % Use a smaller font size for the details
\Who \\ \em % Your name, all text after this will be italicized
%\Where\\ % Your department
\Address\\ % Your address
\CityZip\\ % Your city and zip code
\TEL\\ % Your phone number
\Email\\ % Your email address
%\URL % Your URL
}
\hspace{-1.4in} % Horizontal position of this block, increase to move left, decrease to move right
\vspace{-1in} % Move the letter content up for a more compact look
}

%----------------------------------------------------------------------------------------
%	TO ADDRESS STRUCTURE
%----------------------------------------------------------------------------------------

\def\opening#1{\thispagestyle{empty}
{\centering\fromaddress \vspace{0.6in} \\ % Print the header and from address here, add whitespace to move date down
\hspace*{\longindentation}\today\hspace*{\fill}\par} % Print today's date, remove \today to not display it
{\raggedright \toname \\ \toaddress \par} % Print the to name and address
\vspace{0.4in} % White space after the to address
\noindent #1 % Print the opening line
% Uncomment the 4 lines below to print a footnote with custom text
%\def\thefootnote{}
%\def\footnoterule{\hrule}
%\footnotetext{\hspace*{\fill}{\footnotesize\em Footnote text}}
%\def\thefootnote{\arabic{footnote}}
}

%----------------------------------------------------------------------------------------
%	SIGNATURE STRUCTURE
%----------------------------------------------------------------------------------------

\signature{\Who \What} % The signature is a combination of your name and title

\long\def\closing#1{
\vspace{0.1in} % Some whitespace after the letter content and before the signature
\noindent % Stop paragraph indentation
\hspace*{\longindentation} % Move the signature right
\parbox{\indentedwidth}{\raggedright
#1 % Print the signature text
\vskip 0.65in % Whitespace between the signature text and your name
\fromsig}} % Print your name and title

%----------------------------------------------------------------------------------------

\begin{document}

%----------------------------------------------------------------------------------------
%	TO ADDRESS
%----------------------------------------------------------------------------------------

\begin{letter}
%{Prof. Jones\\
%Mathematics Search Committee\\
%Department of Mathematics\\
%University of California\\
%Berkeley, California 12345\\
{Universidad Carlos III de Madrid
}

%----------------------------------------------------------------------------------------
%	LETTER CONTENT
%----------------------------------------------------------------------------------------

\opening{A quien corresponda,}

Me dirijo a usted para recomendarle a Patricia Hernando Fernández, que ha solicitado la admisión en el Máster Universitario en Ciberseguridad en la Universidad Carlos III de Madrid.

Como Coordinador del Grado de Ingeniería Informática de la Universidad de Burgos conozco a la candidata Patricia Hernando Fernández por sus estudios cursados en el Grado en Ingeniería Informática de la citada universidad. La dedicación y constancia hacia los estudios por parte de la alumna han sido más que notables, siendo la alumna más brillante y con mejor expediente de su promoción. Durante el último curso académico disfrutó de la Beca de Colaboración con Departamentos (del Ministerio que se otorga por expediente académico) y realizó trabajos relacionados con el Aprendizaje Semisupervisado en el Departamento de Ingeniería Informática.

Además, tuve el placer de ser su tutor durante el trabajo de fin de grado titulado \textit{Aprendizaje semisupervisado y ciberseguridad: detección automática de ataques en sistemas de recomendación y phishing} que obtuvo calificación de Matrícula de Honor en la convocatoria de junio del curso 2022-23. El trabajo presentado fue de una calidad técnica y científica más que destacable, habiendo demostrado sus capacidades como ingeniera y futura investigadora. La web diseñada es accesible en \url{https://krini.herokuapp.com/index} y todo el trabajo está públicamente accesible en Github \url{https://github.com/phf1001/semisupervised-learning-in-cibersecurity/tree/main}.

Su interés en la Minería de Datos, la Inteligencia Artificial y la Ciberseguridad ha sido una constante a lo largo de todos sus estudios universitarios, culminando con el previamente mencionado trabajo de fin de grado donde se familiarizó con las técnicas de preprocesamiento y aprendizaje semisupervisado.

Por todos estos motivos, recomiendo a Patricia Hernando Fernández como candidata al Máster en Ciberseguridad por su destacado interés en la materia, alta formación, responsabilidad y eficiencia en el ámbito académico.


\closing{Atentamente,}

%----------------------------------------------------------------------------------------

\end{letter}
\end{document}
